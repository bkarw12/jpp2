\documentclass{article}

\usepackage{polski}
\usepackage[utf8]{inputenc}
\usepackage{indentfirst}
\usepackage{hyperref}

\title{JPP zadanie 2 -- deklaracja języka}
\author{Bartłomiej Karwowski\\385713}
\date{}

\begin{document}

\maketitle

\section{Wstęp}
Opisany w~tym dokumencie język programowania został stworzony 
na podstawie języka \textit{Latte} ze strony:\\

\url{https://www.mimuw.edu.pl/~ben/Zajecia/Mrj2018/Latte/}

\section{Gramatyka języka}
Opis gramatyki w notacji EBNF znajduje się w~pliku \textbf{gram.cf}

\section{Przykładowe programy}
Przykładowe programy znajdują się w~folderze \textbf{prog} w~podfolderach 
\textbf{good} (poprawne programy) oraz \textbf{bad} (niepoprawne programy).
\begin{itemize}
    \item latte1/2/3.in -- przykładowe programy z \textit{Latte}.
    \item *.out -- output przykładowych programów
\end{itemize}

\newpage
\section{Opis języka}
Opis głównej części języka jest analogiczny do \textit{Latte}: \\
,,Program w języku Latte jest listą definicji funkcji. 
Na definicję funkcji składa się typ zwracanej wartości, nazwa, lista argumentów oraz ciało. 
Funkcje muszą mieć unikalne nazwy. 
W~programie musi wystąpić funkcja o~nazwie main zwracająca int 
i~nie przyjmująca argumentów (od niej zaczyna się wykonanie programu).
Funkcje o~typie wyniku innym niż void muszą zwracać wartość za pomocą instrukcji return. Funkcje mogą być wzajemnie rekurencyjne; co za tym idzie mogą być definiowane w~dowolnej kolejności (użycie funkcji może występować przed jej definicją).''

\subsection{Wywołania funkcji i~przesłanianie identyfikatorów}
Wszystkie parametry są przekazywane przez wartość. 
Wewnątrz funkcji parametry formalne zachowują się jak zmienne lokalne.
Przesłanianie identyfikatorów odbywa się ze statycznym ich wiązaniem,
zaś reguły redeklaracji zmiennych są wzorowane na języku C.
Jedynym uchyleniem od tej zasady jest możliwość deklarowania funkcji i~zmiennych o~tej samej nazwie.

\subsection{Zmienne globalne}
Zmienne globalne można deklarować w~dowolnej kolejności (tak jak funkcje).
Oznacza to, że poza ciałem funkcji, na początku działania programu, można przypisać im
tylko stałe wyrażenia (tzn. nie można przypisać jednej zmiennej wartości drugiej).
W~ciele funkcji przypisania są dowolne (tak jak dla zmiennych lokalnych).

\subsection{Instrukcje i wyrażenia}
Wszystkie instrukcje i~wyrażenia które są w~\textit{Latte} 
zachowują się w~tym języku identycznie.

\subsection{Napisy}
Napisy, jak w \textit{Latte}, mogą występować tu jako:
literały, wartości zmiennych, argumentów i~wyników funkcji.
Różnica jest taka, że tutaj napisy jako parametry funkcji przekazywane są przez wartość.
Nie można także konkatenować napisów operatorem +.

\subsection{Predefiniowane funkcje}
W~języku występują następujące predefiniowane funkcje, 
których działanie identyczne jak w~\textit{Latte} (w~przypadku wypisywania outputu
w~dwóch wersjach (odpowiednio -- bez znaku końca linii oraz z~nim):
\begin{itemize}
    \item void printInt(int), void printIntLn(int)
    \item void printString(string), void printStringLn(string)
    \item void error()
\end{itemize}

\subsection{Inne instrukcje}
\begin{itemize}
    \item Pętla for (for i = a to b) -- tak jak w~wymaganiu punktowym nr 7,
    zmienna i~jest \textit{read-only}, 
    zaś wartość b jest wyliczana tylko raz na początku pętli.
    \item break/continue -- analogicznie jak np. w~C++.
\end{itemize}

\subsection{Błędy wykonania}
Język obsługuje następujące błędy wykonania programu:
\begin{itemize}
    \item dzielenie przez 0
    \item wyrażenie modulo 0
    \item wyżej wymieniona funkcja void error()
\end{itemize}

\newpage
\section{Oczekiwana liczba punktów}
Spodziewana ilość punktów za poprawne wykonanie wszystkich poniższych zagadnień -- \textbf{25 pkt.}

\subsection{Na 20 pkt.}
\begin{enumerate}
    \item Co najmniej trzy typy wartości: int, bool i~string.
    \item Literały, arytmetyka, porównania.
    \item Zmienne, operacja przypisania
    \item Jawne wypisywanie wartości na wyjście (instrukcja lub wbudowana procedura print).
    \item while, if (z~else i~bez).
    \item Funkcje lub procedury (bez zagnieżdżania), rekurencja.
    \item b) Zmienne „read-only” i~użycie ich np. w~implementacji pętli for w~stylu Pascala.
    \item Przesłanianie identyfikatorów ze statycznym ich wiązaniem (zmienne lokalne i~globalne).
    \item Obsługa błędów wykonania, np. dzielenie przez zero.
    \item Funkcje przyjmujące i~zwracające wartość dowolnych obsługiwanych typów.
\end{enumerate}

\subsection{Dodatkowe punkty}
\begin{itemize}
    \item 4 pkt. -- Statyczne typowanie (tj. zawsze terminująca faza 
    kontroli typów przed rozpoczęciem wykonania programu).
    \item 1 pkt. -- Operacje przerywające pętlę while -- break i~continue.
\end{itemize}

\end{document}
