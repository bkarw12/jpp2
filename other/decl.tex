\documentclass{article}

\usepackage{polski}
\usepackage[utf8]{inputenc}
\usepackage{indentfirst}
\usepackage{hyperref}

\title{JPP zadanie 2 - deklaracja języka}
\author{Bartłomiej Karwowski\\385713}
\date{}

\begin{document}

\maketitle

\section{Wstęp}

Opisany w tym dokumencie język programowania jest stworzony na podstawie języka \textit{Latte} ze strony:\\

\url{https://www.mimuw.edu.pl/~ben/Zajecia/Mrj2018/Latte/}

\section{Gramatyka języka}

Opis gramatyki w notacji EBNF znajduje się w pliku \textbf{gram.cf}

\section{Przykładowe programy}

Przykładowe programy znajdują się w folderze \textbf{prog}.

\begin{itemize}
    \item latte1/2/3.in - przykładowe programy z \textit{Latte}.
    \item *\_bad.in - przykładowe niepoprawne programy
    \item *.out - output przykładowych programów
\end{itemize}

\newpage
\section{Opis języka}

Program w tym języku (tak jak w \textit{Latte}) jest listą definicji funkcji, a także dodatkowo definicji zmiennych globalnych. Pozostałe części głównej struktury programu są identyczne jak w wymienionym języku, włącznie z wzajemną rekurencją funkcji (zmienne globalne także mogą być deklarowane w dowolnej kolejności).

Wszystkie instrukcje i wyrażenia które są w \textit{Latte} zachowują się w tym języku identycznie. Typy i predefiniowane funkcje także są identyczne. Obsługa parametrów funkcji także jest analogiczna (przekazywanie argumentów przez wartość).

\subsection{Inne instrukcje}

\begin{itemize}
    \item Pętla for (for i = a to b) - tak jak w wymaganiu punktowym nr 7, zmienna i jest \textit{read-only}, zaś wartość b jest wyliczana tylko raz na początku pętli.
    \item break/continue - analogicznie jak np. w C++.
\end{itemize}

\newpage
\section{Oczekiwana liczba punktów}

Spodziewana ilość punktów za poprawne wykonanie wszystkich poniższych zagadnień - \textbf{25 pkt.}

\subsection{Na 20 pkt.}
\begin{enumerate}
    \item Co najmniej trzy typy wartości: int, bool i string.
    \item Literały, arytmetyka, porównania.
    \item Zmienne, operacja przypisania
    \item Jawne wypisywanie wartości na wyjście (instrukcja lub wbudowana procedura print).
    \item while, if (z else i bez).
    \item Funkcje lub procedury (bez zagnieżdżania), rekurencja.
    \item b) Zmienne „read-only” i użycie ich np. w implementacji pętli for w stylu Pascala.
    \item Przesłanianie identyfikatorów ze statycznym ich wiązaniem (zmienne lokalne i globalne).
    \item Obsługa błędów wykonania, np. dzielenie przez zero.
    \item Funkcje przyjmujące i zwracające wartość dowolnych obsługiwanych typów.
\end{enumerate}

\subsection{Dodatkowe punkty}
\begin{itemize}
    \item 4 pkt. - Statyczne typowanie (tj. zawsze terminująca faza kontroli typów przed rozpoczęciem wykonania programu).
    \item 1 pkt. - Operacje przerywające pętlę while - break i continue.
\end{itemize}

\end{document}
